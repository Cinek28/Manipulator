\externaldocument{CH_Wstep}
\externaldocument{CH_Opis_konstrukcji}
\externaldocument{CH_Model}
\externaldocument{CH_EKF}
\externaldocument{CH_Algorytm}

\begin{center}
\textbf{Abstract}
\end{center}
\paragraph{Implementation of Bellman-Ford algorithm for maze-solving by micromouse type robot}
\paragraph{}
The aim of this thesis was solving the shortest path problem for a maze described in the Micromouse competition rules along with the implementation of a chosen algorithm on a mobile robot. The document presents detailed description of project with the quality assessment of applied methods. First chapter shows requirements for Micromouse competition (Chapter 1). Based on the rules, a mechanical and electronic platform of mobile robot was developped (Chapter 2). Next step was to obtain mathematical model of given differential-drive robot (Chapter 3) which was used in chosen approach. These equations are essential in determining the position based on previously known location and measurements acquired from sensors installed on the platform. In order to localize the robot, a SLAM algorithm using Extended Kalman Filter was implemented (Chapter 4). The afore mentioned filter is recommended for estimating non- linear processes and enables the use of sensor fusion. This method gives an optimal information about process state. It is possible to point out where in the maze the robot is. Created map of environment is used in Bellman-Ford algorithm for finding the shortest path through the maze (Chapter 5). The final stage of the project was to conduct experiments in order to assess if the methods used were correct. Based on obtained results the weaknesses and possible improvements were summarized (Chapter 6).
\vspace*{\baselineskip}

\noindent\textbf{Keywords:} \textit{Micromouse, mobile robot, Bellman-Ford algorithm, shortest path problem, Extended Kalman Filter, localization}


